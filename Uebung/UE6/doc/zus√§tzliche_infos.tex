% Created 2019-05-17 Fri 09:23
% Intended LaTeX compiler: pdflatex
\documentclass[11pt]{article}
\usepackage[utf8]{inputenc}
\usepackage[T1]{fontenc}
\usepackage{graphicx}
\usepackage{grffile}
\usepackage{longtable}
\usepackage{wrapfig}
\usepackage{rotating}
\usepackage[normalem]{ulem}
\usepackage{amsmath}
\usepackage{textcomp}
\usepackage{amssymb}
\usepackage{capt-of}
\usepackage{hyperref}
\author{Christoph Viehboeck}
\date{$\backslash$today}
\title{HSC Infos zu UE6}
\hypersetup{
 pdfauthor={Christoph Viehboeck},
 pdftitle={HSC Infos zu UE6},
 pdfkeywords={},
 pdfsubject={},
 pdfcreator={Emacs 27.0.50 (Org mode 9.2.3)}, 
 pdflang={English}}
\begin{document}

\maketitle

\section{Interrupts}
\label{sec:orgb091a5f}

\begin{itemize}
\item Hinzufügen eines Interrupts in den Cordic Core
\item HLS Synthese mit Interrupt
\item SC\textsubscript{Thread} sensitiv auf bool Leitung
\item Interrupt muss auch rückgesetzt werden
\item IRQ Handling
\item Design Erklärung
\end{itemize}

\section{Firmware}
\label{sec:org16c54db}

Handling der Interrupts in der Firmware soll gleich dem des Timers
erfolgen. Daher sollten wie beschrieben die Funktionen implementiert
werden.

\begin{itemize}
\item Aufbau der FW als wenn diese auf HW läuft
\item Alle Strukturen Schnittstellen gleich
\item Basisadressen
\end{itemize}

Einfacher Austausch der API wenn VP fertig ist und auf der HW
verwendet werden kann.

\begin{itemize}
\item Naming: X<instance>\_<function>
\item Error Handling für die Berechnung 
\begin{itemize}
\item Erweiterung Rückgabewert
\item Interrupt wait loop
\end{itemize}
\end{itemize}
\end{document}